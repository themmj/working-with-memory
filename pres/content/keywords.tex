\section{Keywords}
\frame{
	\frametitle{Keywords}
	\framesubtitle{Types}
	
	\begin{description}
		\item[void] Empty data type
		\item[int] Integer with one machine-type word in length (usually 4 bytes, at least 2)
		\item[char]  Character with smallest addressable unit of machine in length (usually 1 byte)
		\item[float] Single precision floating point (usually)
		\item[double] Double precision floating point (usually)
	\end{description}
}

\frame{
	\frametitle{Keywords}
	\framesubtitle{Type Modifiers}
	
	\begin{description}
		\item[short] int: At least 2 bytes length
		\item[long] int: At least 4 bytes length\\
		double: Usually 16 bytes length
		\item[signed] int: Range at least [-32767, +32767]\\
		char: Range at least [-127, +127]
		\item[unsigned] int: Range at least [0, 65535]\\
		char: Range at least [0, 255]
	\end{description}
}

\frame{
	\frametitle{Keywords}
	\framesubtitle{Type Qualifiers}
	
	\begin{description}
		\item[const] Variable value or pointer parameter is unmodifiable
		\item[volatile] Indicates that a variable can be changed unpredictably (e.g. interrupts)
		\item[extern] Indicates that an identifier is defined elsewhere
		\item[static] Preserves variable value after its scope ends (inside executable like globals)
		\item[register] Tells the compiler to store the variable in a CPU register
		\item[auto] Defines a local variable to have a local life time (automatically applied, rarely used)
	\end{description}
}

\frame{
	\frametitle{Keywords}
	\framesubtitle{Flow Control}
	
	\begin{description}
		\item[for] ([expr1]; [expr2]; [expr3]) stmt\\
		expr1 is evaluated before the first iteration\\
		stmt is executed until expr2 evaluates to 0\\
		expr3 is evaluated after every loop
		\item[while] (expr) stmt\\
		stmt is executed until expr evaluates to 0
		\item[do] stmt \key{while} (expr)\\
		stmt is executed until expr evaluates to 0\\
		expr is evaluated after each execution
		\item[if] (expr) stmt [else altstmt]\\
		stmt is executed when expr evaluates to non-0\\
		otherwise altstmt is executed
	\end{description}
}

\frame{
	\frametitle{Keywords}
	\framesubtitle{Flow Control}
	
	\begin{description}
		\item[switch] (expr) stmt\\
		evaluating expr must return int value
		stmt is usually a block of code with arbitrary amount of:\\
		\key{case} constexpr: stmt \\
		constexpr must have unique int value\\
		stmt is executed when evaluated expr equals evaluated constexpr\\
		finally one optional:\\
		\key{default}: defstmt\\
		defstmt is executed when no other branch was taken
	\end{description}
}

\frame{
	\frametitle{Keywords}
	\framesubtitle{Flow Control}
	
	\begin{description}
		\item[continue] causes control to pass to the end of the innermost enclosing \key{while}, \key{do}, \key{for}, \key{switch}
		\item[break] causes control to pass to the statement following the innermost enclosing \key{while}, \key{do}, \key{for}, \key{switch}
		\item[return] [expr]\\
		exits immediately from the currently executing function to the calling routine\\
		optionally returns value of evaluated expr
	\end{description}
}

\frame{
	\frametitle{Keywords}
	\framesubtitle{Spicy Stuff}
	
	\begin{description}
		\item[enum] [tag] \key{\{}name [=value], \key{\}};\\
		defines a set of constants of type \key{int}\\
		value of constant name can be set explicitly or is previous constant's value +1 (first default 0)\\
		tag can be used to declare variables of enum type
		\item[sizeof] expr\\
		returns size in bytes of evaluated expr\\
		(type) can be used to get size of type
		\item[goto] please don't
	\end{description}
}

\frame{
	\frametitle{Keywords}
	\framesubtitle{Spicy Stuff}
	
	\begin{description}
		\item[struct] [structname] \key{\{} { type varnames; } \key{\}} [structvars];\\
		groups variables into a single record\\
		structname is type name of structure\\
		structvars are data definitions\\
		both are optional but one must appear\\
		elements of the record are defined with their type and names\\
		elements are accessed with .
	\end{description}
}

\frame{
	\frametitle{Keywords}
	\framesubtitle{Spicy Stuff}
	
	\begin{description}
		\item[union] [unionname] \key{\{} { type varnames; } \key{\}} [unionvars];\\
		groups variables which share the same storage space\\
		similar to a struct except the variables share the same space\\
		enough space to accommodate for the largest type will be allocated\\
		writing to one element overwrites the others
	\end{description}
}

\frame{
	\frametitle{Keywords}
	\framesubtitle{Spicy Stuff}
	
	\begin{description}
		\item[typedef] definition identifier\\
		creates a new type\\
		assigns the symbol name identifier to the data type definition
	\end{description}
}
